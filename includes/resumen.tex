%!TEX root = ../memoria.tex

Plantilla \LaTeX{} para las Memorias y Tesis del Departamento de Industrias, UTFSM.

Se incluyen también algunos ejemplos de cómo incorporar tablas y gráficos en distintas presentaciones respetando las Normas de Biblioteca para Memorias y Tesis de la UTFSM.

\vspace{10mm}

\paragraph{Palabras Clave.}
\LaTeX{}, Plantilla para Memoria, Departamento de Industrias, UTFSM.

\vspace*{1cm}
\subsection*{¡Importante! [LEAME]}

\subsubsection*{Impresión por un solo lado.}
A partir del año 2016, el Departamento de Industrias sólo requiere la entrega digital de los archivos de memorias y tesis. Por este motivo, este documento está preparado para ser impreso por un solo lado de una hoja (\emph{``oneside''}), y facilitar así su lectura en pantallas. Esta configuración es parte de archivo de clase \inlinecode{thesis_utfsm.cls}.

\subsubsection*{Codificación de caracteres.}

Todos los archivos \inlinecode{*.tex} de esta plantilla han sido preparados ocupando la codificación de caracteres por defecto \emph{unicode} (UTF-8). Windows (y algunas versiones de OSX) ocupan otro tipo de codificación (ej. \emph{Windows-1252} o \emph{Mac Roman}).

Si deseas ocupar esta plantilla y encuentras problemas con los caracteres acentuados, entonces puedes optar por una de estas tres alternativas:
\begin{enumerate}[i)]
    \item cambiar tu editor (TexMaker, TexStudio, TexShop, etc.) para que ocupe UTF-8 como codificación de caracteres por defecto; o
    
    \item cambiar la codificación de cada documento \inlinecode{*.tex} para que ocupe la codificación nativa de tu sistema operativo; y, modificar el archivo \inlinecode{config.tex} la línea que dice:
    
    OSX, Linux: \inlinecode{\\usepackage[utf8x]\{inputenc\}}

    Windows:    \inlinecode{\\usepackage[latin1]\{inputenc\}}
    
    Overleaf:   \inlinecode{\\usepackage[utf8]\{inputenc\}} \url{https://overleaf.com}
    
    \item escribir todo ocupando caracteres pre-acentuados (ej. \inlinecode{\\'a} en lugar de á).
\end{enumerate}

\vspace{10mm}
\begin{framed}
    \noindent\textbf{Recuerde:} 
    
    \noindent Mezclar documentos de distintas codificaciones puede generar muchos problemas al momento de compilar.  
\end{framed}

